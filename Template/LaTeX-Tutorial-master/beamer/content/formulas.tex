\section{Formulas}
\begin{frame}
\frametitle{Formulas}
\framesubtitle{Use Of Mathematical Formulas}

\begin{exampleblock}{New packages in this section}
\begin{multicols}{2}
\begin{itemize}
\item amsmath 
\item amsthm
\item amssymb
\item mathtools
\end{itemize}
\end{multicols}
\end{exampleblock}

\begin{block}{New commands in this section}
\begin{multicols}{2}
\begin{itemize}
\item \color{nounibaredI}\textbackslash sqrt\color{black}\{\}
\item \color{nounibaredI}\textbackslash frac\color{black}\{\}\{\}
\item \color{nounibaredI}\textbackslash int\color{black}\_X
\item \color{nounibaredI}\textbackslash sum\color{black}\_\{\}
\item \color{nounibaredI}\textbackslash lim\color{black}\_\{\}
\item \color{nounibaredI}\textbackslash prod\color{black}
\item \color{nounibaredI}\textbackslash limits\color{black}\_\{\}
\item \color{nounibaredI}\textbackslash dots\color{black}
\item \color{nounibaredI}\textbackslash cdot\color{black}
\item \color{nounibaredI}\_\color{black}
\item \color{nounibaredI}\^~\color{black}
\end{itemize}
\end{multicols}
\end{block}

\end{frame}

%-------------------------------------------------------------------------------
\begin{frame}
\frametitle{Formulas}
\framesubtitle{\ldots ~A Marvel Of Beauty In \LaTeX !}

\begin{columns}
\begin{column}{.3\textwidth}
{\huge $2 \sqrt{\frac{\pi ^2}{3}\cdot c_{2}}$}
\end{column}

\begin{column}{.7\textwidth}
	$\underbrace{
	\color{unibayellowI}\text{\$}
	\color{black}2
	\color{nounibaredI}\backslash \text{sqrt}
	\color{black}\{
	\color{nounibaredI}\backslash \text{frac}
	\color{black}\{
	\color{nounibaredI}\backslash \text{pi}\color{nounibaredI}
	~\hat{}~\color{black}2\}\{3\color{black}\}
	\color{nounibaredI}\backslash
	\color{nounibaredI}\text{cdot}~
	\color{black} \text{c}
	\color{nounibaredI}\_
	\color{black}2\}
	\color{unibayellowI}\text{\$}
}$
\color{black}

The formula-environment begins and ends with \color{unibayellowI}\$ \color{black} .

\medskip
	$\underbrace{
	\color{nounibaredI}\backslash \text{sqrt}
	\color{black}\{
	\color{nounibaredI}\backslash \text{frac}
	\color{black}\{
	\color{nounibaredI}\backslash \text{pi}\color{nounibaredI}
	~\hat{}~\color{black}2\}\{3\color{black}\}
	\color{nounibaredI}\backslash
	\color{nounibaredI}\text{cdot}~
	\color{black} \text{c}
	\color{nounibaredI}\_
	\color{black}2\}
}$
\color{black}


This whole part is the radical.

\bigskip
$\underbrace{
	\color{nounibaredI}\backslash \text{frac}
	\color{black}\{
	\color{nounibaredI}\backslash \text{pi}\color{nounibaredI}
	~\hat{}~\color{black}2\}\{3\color{black}\}
}$

A fraction has always a numerator and a denominator.
\end{column}
\end{columns}
\end{frame}

%-------------------------------------------------------------------------------

\begin{frame}
\frametitle{Formulas}
\framesubtitle{\ldots ~A Marvel Of Beauty In \LaTeX !}
\begin{columns}
	\begin{column}{.4\textwidth}
		\flushright
		$\int_0^\infty$
	\end{column}
	\begin{column}{.6\textwidth}
		\flushleft
		{\ttfamily\color{unibayellowI}\$\color{nounibaredI}\textbackslash\color{nounibaredI}int\_\color{black}0\color{nounibaredI}\textasciicircum \textbackslash infty\color{unibayellowI}\$}
	\end{column}
\end{columns}
\begin{columns}
	\begin{column}{.4\textwidth}
		\flushright
		$\sum_{i=1}^n$
	\end{column}
	\begin{column}{.6\textwidth}
		\flushleft
		{\ttfamily \color{unibayellowI}\$\color{nounibaredI}\textbackslash
			\color{nounibaredI}sum\_\color{black}\{i=1\}\color{nounibaredI}\textasciicircum
			\color{black}n\color{unibayellowI}\$}
	\end{column}
\end{columns}

\begin{columns}
	\begin{column}{.4\textwidth}
		\flushright
		$\lim_{n \rightarrow \infty}$
	\end{column}
	\begin{column}{.6\textwidth}
		\flushleft
		{\ttfamily \color{unibayellowI}\$\color{nounibaredI}\textbackslash
			\color{nounibaredI}lim\_\color{black}\{n \color{nounibaredI}\textbackslash
			\color{nounibaredI}rightarrow \color{nounibaredI}\textbackslash infty\color{black}\}\color{unibayellowI}\$}
	\end{column}
\end{columns}

\begin{columns}
	\begin{column}{.4\textwidth}
		\flushright
		$\prod\limits_{i=1}^{n+1}i = 1 \cdot 2 \cdot \ldots \cdot n \cdot (n+1)$
	\end{column}
	\begin{column}{.6\textwidth}
		\flushleft
		{\ttfamily \color{unibayellowI}\$%
			\color{nounibaredI}\textbackslash\color{nounibaredI}prod\textbackslash  limits\_\color{black}\{i=1\}\color{nounibaredI}\^{}\color{black}\{n+1\} i = 1 \color{nounibaredI}\textbackslash \color{nounibaredI}cdot \color{black}2 \color{nounibaredI}\textbackslash \color{nounibaredI}cdot \color{nounibaredI}\textbackslash \color{nounibaredI}ldots \color{nounibaredI}\textbackslash\color{nounibaredI}cdot \color{black}n \color{nounibaredI}\textbackslash \color{nounibaredI}cdot \color{black}(n+1)\color{unibayellowI}\$}
	\end{column}
\end{columns}
\bigskip
The American Mathematical Society has a wonderful guide for the {\ttfamily amsmath}-package.\footnote{ftp://ftp.ams.org/pub/tex/doc/amsmath/amsldoc.pdf}
\end{frame}