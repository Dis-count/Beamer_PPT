\color{nounibaredI}\color{nounibaredI}\textbackslash documentclass\color{black}\color{nounibagreenI}[a4paper, pdftex, 12pt, ngerman]\color{black}\{article\} \\
\color{nounibaredI}\color{nounibaredI}\textbackslash usepackage\color{black}\color{nounibagreenI}[utf8]\color{black}\{inputenc\} \\
\color{nounibaredI}\color{nounibaredI}\textbackslash usepackage\color{black}\color{nounibagreenI}[T1]\color{black}\{fontenc\} \\
\color{nounibaredI}\color{nounibaredI}\textbackslash usepackage\color{black}\{babel\} \\
\color{nounibaredI}\color{unibablueI}\textbackslash\color{unibablueI}begin\color{black}\color{black}\{document\} \\
The footnote\color{nounibaredI}\color{nounibaredI}\textbackslash footnote\color{black}\{here comes the footnote text\} to a word or a text always appears on the page it belongs to. The footnote text is bracketed.\color{nounibaredI}\color{nounibaredI}\textbackslash \color{nounibaredI}\textbackslash \color{black} \\
\color{nounibaredI}\color{nounibaredI}\textbackslash \color{nounibaredI}\textbackslash \color{black} \\
A manual numbering is possible, too.\color{nounibaredI}\color{nounibaredI}\textbackslash footnote\color{black}\color{nounibagreenI}[10]\color{black}\{just as this\}, even without footnotetext\color{nounibaredI}\color{nounibaredI}\textbackslash footnotemark\color{black}\color{nounibagreenI}[2]\color{black}. \\
\color{nounibaredI}\color{unibablueI}\textbackslash\color{unibablueI}end\color{black}\color{black}\{document\} \\
