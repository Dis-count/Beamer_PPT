\section{Footnotes}

\begin{frame}
\frametitle{Footnotes}
\framesubtitle{Use Of Footnotes}
\begin{block}{New commands in this section}
\begin{itemize}
\item \color{nounibaredI}\textbackslash footnote
\item \textbackslash footnotemark\color{nounibagreenI}[]\color{black}
\end{itemize}
\end{block}
\end{frame}

%-------------------------------------------------------------------------------

\begin{frame}
\frametitle{Footnotes}
\framesubtitle{Commands}
\begin{columns}
\begin{column}{0.5\textwidth}
\begin{ttfamily}\scriptsize
\color{nounibaredI}\color{nounibaredI}\textbackslash documentclass\color{black}\color{nounibagreenI}[a4paper, pdftex, 12pt, ngerman]\color{black}\{article\} \\
\color{nounibaredI}\color{nounibaredI}\textbackslash usepackage\color{black}\color{nounibagreenI}[utf8]\color{black}\{inputenc\} \\
\color{nounibaredI}\color{nounibaredI}\textbackslash usepackage\color{black}\color{nounibagreenI}[T1]\color{black}\{fontenc\} \\
\color{nounibaredI}\color{nounibaredI}\textbackslash usepackage\color{black}\{babel\} \\
\color{nounibaredI}\color{unibablueI}\textbackslash\color{unibablueI}begin\color{black}\color{black}\{document\} \\
The footnote\color{nounibaredI}\color{nounibaredI}\textbackslash footnote\color{black}\{here comes the footnote text\} to a word or a text always appears on the page it belongs to. The footnote text is bracketed.\color{nounibaredI}\color{nounibaredI}\textbackslash \color{nounibaredI}\textbackslash \color{black} \\
\color{nounibaredI}\color{nounibaredI}\textbackslash \color{nounibaredI}\textbackslash \color{black} \\
A manual numbering is possible, too.\color{nounibaredI}\color{nounibaredI}\textbackslash footnote\color{black}\color{nounibagreenI}[10]\color{black}\{just as this\}, even without footnotetext\color{nounibaredI}\color{nounibaredI}\textbackslash footnotemark\color{black}\color{nounibagreenI}[2]\color{black}. \\
\color{nounibaredI}\color{unibablueI}\textbackslash\color{unibablueI}end\color{black}\color{black}\{document\} \\

\end{ttfamily}
\end{column}
\begin{column}{0.5\textwidth}
\begin{ttfamily}\color{nounibaredI}\textbackslash footnote\color{black}\{footnotetext\}\end{ttfamily} Creates a footnote at this position with
automatical numbering.

\begin{ttfamily}\color{nounibaredI}\textbackslash footnote\color{nounibagreenI}[number]\color{black}\{footnotetext\}\end{ttfamily} A manual
numbering is possible, too.

\begin{ttfamily}\color{nounibaredI}\textbackslash footnotemark\color{nounibagreenI}[number]\color{black}\end{ttfamily}
It is also possible to use a number without a footnotetext.
\end{column}
\end{columns}
\bigskip
The numbering is automatical and continous, independent of 
a new page or {\ttfamily section}.
\end{frame}